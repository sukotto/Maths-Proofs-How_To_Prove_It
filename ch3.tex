\documentclass{article}
\begin{document}
\title{Chapter 3 - Proofs}
\author{Scott Brown}
\maketitle
\section*{3.1. Proof Strategies}
1. \textbf{Theorem.} \textit{Suppose n is an integer larger than 1 and n is not prime. Then $2^n - 1$ is not prime.}\\
(a) Hypotheses: \\
 n is an integer larger than 1\\
n is not prime\\
 Conclusion:\\
  $2^n - 1$ is not prime\\
  When $n = 6$ both the hypotheses are true, because $6 > 1$ and $6$ is not prime ($2*3=6$).\\ The theorem says that $2^6-1$ is not prime, and this is correct since 
  $2^6-1=63$ and $63$ is not prime ($9*7=63$).\\
  \linebreak
  (b) For the case $n = 15$, $n > 1$ and $n$ is not prime ($15 = 3*5$), so the hypotheses are true. The conclusion is also true because $2^{15}-1 = 32767$ and $32767$ is not prime ($32767 = 151 * 31 * 7$).\\
  \linebreak
  (c) For the case $n = 1$, $11 > 1$ and $11$ is prime. Because one of the hypotheses is false, nothing can be concluded from the theorem in this case.\\
  \linebreak
 2. \textbf{Theorem.} \textit{Suppose that $b^2 > 4ac$. Then the quadratic equation $ax^2 + bx + c = 0$ has exactly two real solutions.}\\
 (a) Hypothesis: $b^2 > 4ac$
     Conclusion: The quadratic equation $ax^2 + bx + c =0$ has exactly two real solutions.\\
     \linebreak
 (b) To give an instance of the theorem, values of a,b and c must be specified, but not x. This is because the values of a, b and c must be known to decide if the hypothesis is true. The value of x is unspecified because the theorem applies to all possible values of x.\\
 \linebreak
 (c)$a=2, b=5, c=3$\\
 $5^2 > 4 * 2 * 3 \equiv 25 > 24$ which is true, so the hypothesis holds. \\
 $2x^2 + 5x + 3 = 0$\\
 $2 * 3 = 6$\\
 $2 + 3 = 5$\\
 $2x^2 + 2x + 3x + 3$\\
 $2x(x + 1) + 3 (x + 1)$\\
 $(2x + 3)(x + 1) = 0$\\
 $x = -1$ or $x = -3/2$ so there are exactly two real solutions.\\ 
 The conclusion is correct.\\
 \linebreak
 (d)$a = 2, b = 4, c = 3$\\
 $4^2 > 4 * 2 * 3 \equiv 16 > 24$ which is false. Since the hypothesis is false, nothing can be concluded from the theorem in this case.\\
 \linebreak
 3. \textbf{Incorrect Theorem.} \textit{Suppose n is a natural number larger than 2, and n is not a prime number. Then $2n + 13$ is not a prime number. }\\
 Hypotheses: n is a natural number larger than 2\\
             n is not a prime number\\
 Conclusion: $2n + 13$ is not a prime number\\
 Counterexample: $n = 8$\\
 $8 > 2$ and 8 is not prime, so the hypotheses are true.\\
 $2 * 8 + 13 = 29$ which is prime, so the conclusion is false.\\
 The hypotheses are true, but the conclusion is false, so the theorem is incorrect.\\
 \linebreak
4. \textit{Proof.} Suppose $ 0 < a < b $. Then $b - a > 0$.\\
$(b - a)(b + a) > 0 * (b + a)$\\
$b^2 + ab - ab - a^2 > 0$\\
$b^2 - a^2 > 0$\\
Since $b^2 - a^2 > 0$, it follows that $a^2 < b^2$. Therefore if $0 < a < b$ then $a^2 < b^2$.\\
\linebreak
5. Suppose $a$ and $b$ are real numbers. $a < b < 0 \Rightarrow a^2 > b^2$\\
A negative number multiplied by a negative is positive, and both $a$ and $b$ are negative.\\
Multiply by a: $a^2 > ab$\\
Multiply by b: $ab > b^2$\\
$a^2 > ab > b^2$, therefore if $a < b < 0$ then $a^2 > b^2$.\\
\linebreak
6. Suppose $a$ and $b$ are real numbers. If $0 < a < b$ then $1/b < 1/a$. \\
Divide both sides by a:\\
$a < b \equiv a/a < b/a \equiv 1 < b/a $\\
Then divide both sides by b:\\
$1/b < b/(ab) \equiv 1/b < 1/a$\\
Therefore if $0 < a < b$ then $1/b < 1/a$\\
\linebreak
\end{document}